%\begin{abstractzh}
% TODO 中文摘要
% \end{abstractzh}


\begin{abstracten}
% 1. Current Situation
% 2. Problem Statement
% 3. Proposed Solution
% 4. Solution Details
% 5. Results and Contributions

Smartwatches, wearable electronics, and other miniature devices are becoming increasingly popular. However, text input on such small devices remains a challenge due to small form factors, especially for non-Latin languages that require more complex key entry techniques. We implement and compare 3 fully functional soft keyboards for typing Mandarin Chinese using the Hanyu Pinyin system on the latest generation of circular smartwatch devices. We introduce 2 such novel adaptive keyboards called Growing Finals and Pinyin Syllables,  which change dynamically based on the current input and the limited number of possible subsequent letters, optimize available screen size, and improve better user experience on small screens.
Our evaluation is based on a user study with 15 participants and shows input speeds of around 19.4 CCPM for Growing Finals and 18.5 CCPM for Pinyin Syllables after 20 minutes of practice. More than half of the participants preferred one of these input methods over the standard QWERTY keyboard.
% We discovered from participants' keyboard activites that over half preferred one of our proposed novel input methods versus standard QWERTY input due to lesser frustration and fewer error encounters. 
\end{abstracten}


\begin{comment}
\category{H.5.2.}{Information Interfaces and Presentation (e.g. HCI)}{User Interfaces; Input devices and strategies (e.g., mouse, touchscreen)}

\terms{Text Entry, Pinyin, Smartwatch, Human Factors, Performance}

\keywords{Chinese text entry; smartwatch; adaptive keyboard}
\end{comment}
